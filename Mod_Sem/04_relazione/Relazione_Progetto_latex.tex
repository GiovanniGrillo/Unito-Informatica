\documentclass[12pt]{article}
\usepackage[utf8]{inputenc}
\usepackage{tcolorbox}
\usepackage{amsmath}
\usepackage{hyperref}
\usepackage{geometry}
\usepackage{xcolor}
\usepackage{graphicx}
\geometry{margin=2.5cm}

\begin{document}

\thispagestyle{empty}

\begin{center}
    {\large \textsc{Università degli Studi di Torino}}\\
    {\large \textsc{Dipartimento di Informatica}}\\[0.8cm]
    {\textcolor{gray}{\large\href{https://informatica.i-learn.unito.it/course/view.php?id=3571}
    {Modellazione Concettuale del Web Semantico — a.a.\ 2025/2026}}}\\[2.2cm]
    {\Huge \textbf{Relazione del Progetto}}\\[0.8cm]
    {\LARGE \textbf{Narrative Universes Ontology}}\\[1.4cm]

    {\textcolor{gray}{\large Alessandro Olivero (matricola: 915069)\\
    Giovanni Grillo (matricola: 989819)}}
\end{center}

\vfill

\begin{center}
    {\textcolor{gray}{Gennaio 2026}}
\end{center}

\newpage

% =================================================================
% PARTE 1: INTRODUZIONE E MOTIVAZIONI
% =================================================================

\section*{Introduzione e Motivazioni}

Il progetto sviluppa un'ontologia OWL per la rappresentazione formale di universi narrativi transmediali (Harry Potter, Percy Jackson, Il Signore degli Anelli). L'ontologia modella personaggi, luoghi, opere, organizzazioni e le loro relazioni semantiche, ed è interrogabile tramite SPARQL con integrazione di dati esterni da Wikidata.

\textbf{Motivazione:} I franchise narrativi contemporanei si espandono attraverso molteplici media (libri, film, serie TV), generando complessità informativa. I dati sono frammentati su fonti eterogenee (wiki fan-made, database cinematografici) senza integrazione semantica. Un'ontologia formale permette di: (1) strutturare la conoscenza narrativa in modo coerente, (2) effettuare analisi comparative tra universi, (3) integrare dati locali con Linked Open Data, (4) automatizzare classificazioni tramite reasoning.

Dal punto di vista \textbf{culturale}, il progetto sistematizza la conoscenza narrativa contemporanea. Dal punto di vista \textbf{tecnico}, dimostra l'applicabilità del Web Semantico a domini complessi in evoluzione.

% =================================================================
% PARTE 2: REQUIREMENTS
% =================================================================

\section*{Requirements dell'Ontologia}

\subsection*{Finalità}
\begin{itemize}
    \item \textbf{Catalogazione}: repository formale di universi narrativi con personaggi, luoghi, opere
    \item \textbf{Analisi narratologica}: esplorazione di pattern (protagonisti, antagonisti, mentori, alleanze)
    \item \textbf{Interoperabilità}: allineamento con schema.org e Wikidata
    \item \textbf{Estensibilità}: aggiunta di nuovi universi senza modifiche strutturali
\end{itemize}

\subsection*{Utenti Target}
\textbf{Appassionati}: informazioni dettagliate su personaggi e trame. \textbf{Studiosi di narratologia}: analisi di archetipi e strutture ricorrenti. \textbf{Sviluppatori}: backend semantico per applicazioni web. \textbf{Ricercatori}: studio di modellazione ontologica e query federate.

\subsection*{Task Supportati}
Ricerca personaggi per tipo e universo; esplorazione relazioni (alleanze, nemicizie, mentoring); confronto opere (libri vs adattamenti); tracciamento apparizioni cross-mediali; integrazione dati Wikidata; inferenza automatica tramite SWRL.

% =================================================================
% PARTE 3: DESCRIZIONE DEL DOMINIO
% =================================================================

\section*{Descrizione del Dominio}

Gli universi narrativi transmediali sono sistemi narrativi che si sviluppano attraverso molteplici opere (libri, film, serie TV, videogiochi), mantenendo coerenza interna di personaggi, luoghi e trame. Esempi: \textit{Harry Potter} (J.K. Rowling), \textit{Percy Jackson} (Rick Riordan), \textit{Il Signore degli Anelli} (J.R.R. Tolkien).

\textbf{Entità principali:}
\begin{itemize}
    \item \textbf{Personaggi}: protagonisti, antagonisti, mentori, alleati con ruoli narrativi specifici
    \item \textbf{Luoghi}: ambientazioni fisiche (Hogwarts, Campo Mezzosangue, Shire) e spazi liminali
    \item \textbf{Opere}: libri, film, serie TV con relazioni di adattamento e serialità
    \item \textbf{Organizzazioni}: scuole, ordini, fazioni a cui appartengono i personaggi
    \item \textbf{Oggetti}: artefatti magici con poteri specifici
    \item \textbf{Abilità}: poteri magici, divini o simbolici posseduti dai personaggi
\end{itemize}

\textbf{Relazioni significative:} appartenenza a universi, apparizioni in opere, alleanze e nemicizie tra personaggi, possesso di oggetti, concessione di abilità, relazioni di mentoring, ambientazione geografica.

\textbf{Fonti documentali:} Harry Potter Wiki (\url{https://harrypotter.fandom.com}), Riordan Wiki (\url{https://riordan.fandom.com}), Tolkien Gateway (\url{http://tolkiengateway.net}), Wikidata, IMDb, Goodreads.

% =================================================================
% PARTE 4: COMPETENCY QUESTIONS
% =================================================================

\section*{Competency Questions}

Le seguenti domande guidano la progettazione e validano la completezza dell'ontologia:

\begin{enumerate}
    \item \textbf{CQ1 (Ruoli narrativi)}: Quali sono i protagonisti/antagonisti/mentori dell'universo X?
    \item \textbf{CQ2 (Relazioni sociali)}: Chi sono gli alleati e i nemici del personaggio X?
    \item \textbf{CQ3 (Possedimenti)}: Quali personaggi possiedono artefatti di potere?
    \item \textbf{CQ4 (Abilità)}: Quali abilità magiche/divine possiede il personaggio X?
    \item \textbf{CQ5 (Opere e pubblicazioni)}: Quali opere sono state pubblicate dopo il 2000?
    \item \textbf{CQ6 (Valutazioni)}: Quali opere hanno rating superiore a 9.0?
    \item \textbf{CQ7 (Adattamenti)}: Quali libri hanno adattamenti cinematografici?
    \item \textbf{CQ8 (Organizzazioni)}: Quali personaggi appartengono all'organizzazione X?
    \item \textbf{CQ9 (Integrazione LOD)}: Chi è il regista del film X secondo Wikidata?
    \item \textbf{CQ10 (Inferenze SWRL)}: Quali personaggi sono classificati come "antichi" o "potenti"?
\end{enumerate}

% =================================================================
% PARTE 5: DOCUMENTAZIONE DEL DOMINIO
% =================================================================

\newpage

\section*{Documentazione del Dominio}

\subsection*{Allineamenti con Ontologie Standard}

L'ontologia è allineata con \textbf{Wikidata} (tramite SKOS).

\subsubsection*{Wikidata (SKOS)}
\begin{tcolorbox}[colback=white,colframe=black,title=\textbf{Allineamenti Wikidata},arc=2mm,boxrule=0.8pt]
\begin{itemize}
    \item \texttt{Movie} \texttt{skos:closeMatch} \texttt{wd:Q11424}
    \item \texttt{Character} \texttt{skos:closeMatch} \texttt{wd:Q95074}
    \item \texttt{Location} \texttt{skos:closeMatch} \texttt{wd:Q17334923}
    \item \texttt{NarrativeUniverse} \texttt{skos:closeMatch} \texttt{wd:Q55983715}
\end{itemize}
\end{tcolorbox}

\textbf{Motivazione:} Uso di \texttt{skos:closeMatch} per corrispondenza approssimativa senza dipendenze logiche rigide. Essenziale per query federate SPARQL.

\subsection*{Pattern di Modellazione OWL}

\subsubsection*{1. Classi Enumerate}
\begin{tcolorbox}[colback=white,colframe=black,title=\textbf{Enumerazioni},arc=2mm,boxrule=0.8pt]
\textbf{OriginType} $\equiv$ \{HumanOrigin, DivineOrigin, GiantOrigin, ElvenOrigin, SpiritOrigin\}

\textbf{PowerType} $\equiv$ \{MagicalPower, DivinePower, CorruptivePower, SymbolicPower, OrdinaryPower\}
\end{tcolorbox}

\textbf{Esempi:} Percy Jackson ha \texttt{DivineOrigin} (figlio di Poseidone); Anello Unico ha \texttt{CorruptivePower}.

\subsubsection*{2. Restrizioni OWL}
\begin{tcolorbox}[colback=white,colframe=black,title=\textbf{Restrizioni},arc=2mm,boxrule=0.8pt]
\begin{itemize}
    \item \textbf{PowerArtifact} $\equiv$ Object $\sqcap$ $\exists$grantsAbility.Ability
    \item \textbf{HybridCharacter} $\equiv$ Character $\sqcap$ $\geq$2 hasOriginType
    \item \textbf{Antagonist} $\equiv$ Character $\sqcap$ $\exists$enemyOf.Protagonist
\end{itemize}
\end{tcolorbox}

\textbf{Funzione:} Classificazione automatica via reasoner. Se un personaggio ha \texttt{enemyOf} un protagonista, viene classificato automaticamente come \texttt{Antagonist}.

\subsubsection*{3. Property Chains}
\begin{tcolorbox}[colback=white,colframe=black,title=\textbf{Property Chains},arc=2mm,boxrule=0.8pt]
\begin{itemize}
    \item \textbf{indirectlyMentors} $\sqsubseteq$ mentors $\circ$ alliedWith
    \item \textbf{indirectlySetIn} $\sqsubseteq$ setIn $\circ$ partOf
\end{itemize}
\end{tcolorbox}

\subsubsection*{4. Proprietà Speciali}
\textbf{Inverse:} \texttt{appearsIn} $\leftrightarrow$ \texttt{features}. \textbf{Transitive:} \texttt{partOf}, \texttt{sequelOf}. \textbf{Functional:} \texttt{hasMainProtagonist}. \textbf{Symmetric:} \texttt{alliedWith}.

\subsection*{Regole SWRL}

5 regole implementate con Drools per inferenze quantitative impossibili in OWL DL:

\subsubsection*{Regola 1: Personaggi Antichi}
\texttt{Personaggio(?x) $\wedge$ AnnoDiNascita(?x, ?year) $\wedge$ swrlb:lessThan(?year, 1900) $\rightarrow$ PersonaggioAntico(?x)}

\textbf{Built-in:} \texttt{swrlb:lessThan}. \textbf{Esempi:} Albus Dumbledore (1881), Aberforth Dumbledore (1884).

\subsubsection*{Regola 2: Opere Moderne}
\texttt{OperaNarrativa(?w) $\wedge$ AnnoDiPubblicazione(?w, ?year) $\wedge$ swrlb:greaterThan(?year, 2000) $\rightarrow$ OperaModerna(?w)}

\textbf{Esempi:} Pietra Filosofale Film (2001), Ladro di Fulmini Libro (2005).

\subsubsection*{Regola 3: Protagonisti Potenti}
\texttt{Protagonista(?p) $\wedge$ Possiede(?p, ?artifact) $\wedge$ ArtefattoDiPotere(?artifact) $\rightarrow$ ProtagonistaPotente(?p)}

\textbf{Esempi:} Harry Potter (Mantello Invisibilità), Percy Jackson (Riptide), Frodo (Anello).

\subsubsection*{Regola 4: Opere di Alto Valore}
\texttt{OperaNarrativa(?w) $\wedge$ Valutazione(?w, ?r) $\wedge$ swrlb:greaterThan(?r, 9.0) $\rightarrow$ OperaDiAltoValore(?w)}

\textbf{Built-in su float.} \textbf{Esempi:} Compagnia dell'Anello Libro (9.8), Pietra Filosofale Libro (9.6).

\subsubsection*{Regola 5: Alleanze Indirette}
\texttt{Personaggio(?x) $\wedge$ Personaggio(?y) $\wedge$ Personaggio(?z) $\wedge$ AlleatoCon(?x, ?y) $\wedge$ AlleatoCon(?y, ?z) $\rightarrow$ IndirettamenteAlleatoCon(?x, ?z)}

\textbf{Funzione:} Transitività alleanze. \textbf{Esempio:} Harry $\rightarrow$ Hermione $\rightarrow$ Ron $\Rightarrow$ Harry $\rightarrow$ Ron (indiretto).

\textbf{Esecuzione:} Regole eseguite con motore Drools, inferenze materializzate permanentemente nell'A-Box.

\subsection*{Dati di Esempio}

\subsubsection*{Personaggi}
\begin{itemize}
    \item \textbf{Harry Potter} (Protagonista): nato 1980, alleato con Ron e Hermione, nemico di Voldemort, possiede Mantello Invisibilità e Bacchetta Agrifoglio, ha abilità Patronus e Parseltongue, appartiene a Gryffindor, appare in 7 libri e 8 film.
    \item \textbf{Percy Jackson} (Protagonista): figlio di Poseidone (DivineOrigin), possiede Riptide, ha abilità di controllo acqua, appartiene a Campo Mezzosangue.
    \item \textbf{Frodo Baggins} (Protagonista): hobbit, possiede Anello Unico (CorruptivePower), alleato con Compagnia dell'Anello.
\end{itemize}

\subsubsection*{Luoghi}
\begin{itemize}
    \item \textbf{Hogwarts} (LuogoSicuro, Scuola): ospita 4 case, sede di eventi narrativi principali, parte del Regno Unito magico.
    \item \textbf{Binario 9¾} (SpazioLiminale): ingresso al mondo magico, esempio di schema:DefinedRegion.
\end{itemize}

\subsubsection*{Opere}
\begin{itemize}
    \item \textbf{Harry Potter e la Pietra Filosofale} (Libro): pubblicato 1997, rating 9.6, 309 pagine, adattato in film 2001.
    \item \textbf{La Compagnia dell'Anello} (Libro): pubblicato 1954, rating 9.8, sequel di Lo Hobbit.
\end{itemize}

% =================================================================
% PARTE 6: LODE
% =================================================================

\section*{Documentazione Automatica (LODE)}

L'ontologia è stata documentata con \textbf{LODE} (Live OWL Documentation Environment): \url{https://essepuntato.it/lode/}

Il documento generato (\texttt{Narrative-Universes-Ontology-LODE.html} o anche .pdf allegato) fornisce elenco completo di classi, proprietà, individui con descrizioni e gerarchie.

% =================================================================
% PARTE 7: VISUALIZZAZIONI
% =================================================================

\newpage

\section*{Visualizzazioni}

\subsection*{Tassonomia Classi Principali}

\begin{verbatim}
owl:Thing
├─ Abilità
│  ├─ Abilità Divina
│  └─ Abilità Magica
├─ Luogo
│  ├─ Luogo Sicuro
│  ├─ Spazio Liminale
│  └─ Zona Pericolosa
├─ Oggetto
│  ├─ Arma
│  └─ ArtefattoDiPotere
├─ OperaNarrativa
│  ├─ Film
│  ├─ Libro
│  ├─ OperaMaggiore
│  ├─ OperaModerna
│  └─ Serie TV
├─ Organizzazione
│  └─ Scuola
├─ Personaggio
│  ├─ Alleato
│  ├─ Antagonista
│  ├─ Mentore
│  ├─ Personaggio Ibrido
│  ├─ Personaggio Non Umano
│  ├─ Personaggio Umano
│  ├─ PersonaggioAntico
│  └─ Protagonista
│     └─ ProtagonistaPotente
├─ Tipo di Origine
├─ Tipo di Potere
└─ Universo Narrativo
\end{verbatim}
\newpage
\subsection*{Template con Esempi}
\textbf{Template Personaggio Protagonista:}
\begin{verbatim}
:'Harry Potter' rdf:type :Protagonista, :PersonaggioUmano, :ProtagonistaPatente ;
    rdfs:label "Harry Potter"@it ;
    rdfs:comment "Il protagonista della saga, il ragazzo che è sopravvissuto"@it ;
    :AlleatoCon 'Hermione Granger', 'Ron Weasley' ;
    :appareIn 'Harry Potter e la Pietra Filosofale' ;
    :appartieneAUniverso 'Universo di Harry Potter' ;
    :haAbilità Serpentese .
\end{verbatim}

\textbf{Template Opera con Adattamento:}
\begin{verbatim}
:'Harry Potter e la Pietra Filosofale' rdf:type :Libro, :OperaMaggiore ;
    rdfs:label "Harry Potter e la Pietra Filosofale"@it ;
    rdfs:comment "Il primo libro della saga di Harry Potter"@it ;
    :numeroDiPagine "309"^^xsd:integer ;
    :appartieneAUniverso 'Universo di Harry Potter' ;
    :haProtagonistaprincipale 'Harry Potter' ;
    :ambientatoIn Hogwarts .
\end{verbatim}

\subsection*{Tabella Triple Significative}

\begin{table}[h]
\centering
\small
\begin{tabular}{|l|l|l|}
\hline
\textbf{Subject} & \textbf{Types/ Property assertions} & \textbf{Object} \\
\hline
HarryPotter & rdf:type & Protagonista \\
HarryPotter & rdf:type & ProtagonistaPotente \\
HarryPotter & AlleatoCon & RonWeasley \\
HarryPotter & possiede & Mantello dell'Invisibilità \\
InvisibilityCloak & rdf:type & ArtefattoDiPotere \\
Hogwarts & rdf:type & Luogo Sicuro \\
Hogwarts & ospita & Scuola di Hogwarts \\
\hline
\end{tabular}
\caption{Esempio di triple nell'ontologia}
\end{table}

% =================================================================
% PARTE 8: A-BOX E INFERENZE
% =================================================================

\section*{A-Box e Inferenze}
\textbf{Popolamento:} 72 individui completi distribuiti su 3 universi (Harry Potter: 30, Percy Jackson: 15, LOTR: 15). \textbf{Consistenza:} Verificata con reasoner Drools, nessuna contraddizione rilevata. \textbf{Inferenze proposte:} PersonaggioAntico (2 individui: Dumbledore, Aberforth), OperaModerna (15 individui, tutte opere post-2000), ProtagonistaPotente (3 individui: Harry, Percy, Frodo), OperaDiAltoValore (8 individui con rating > 9.0), IndirettamenteAlleatoCon (12 nuove relazioni inferite).

% =================================================================
% PARTE 9: QUERY SPARQL E APPLICAZIONE
% =================================================================

\newpage

\section*{Query SPARQL e Applicazione Client}
\subsection*{Query Implementate}
13 query SPARQL sviluppate per supportare l'interazione utente con graphDB, di cui 3 query che attingono verso Wikidata con clausola \texttt{SERVICE} per l'integrazione di dati esterni sui film. Le query federate arricchiscono i dati locali con informazioni autorevoli provenienti da Wikidata (registi, date di uscita precise, durate), combinando tramite \texttt{UNION} i risultati del repository locale GraphDB con Wikidata.

\begin{table}[h]
\centering
\small
\begin{tabular}{|p{5cm}|p{6.5cm}|p{3cm}|}
\hline
\textbf{Query} & \textbf{Funzione} & \textbf{Features SPARQL} \\
\hline
getUniverses & Lista universi con conteggi di personaggi, luoghi e opere & COUNT, GROUP BY \\
\hline
getUniverseDetails & Dettagli e statistiche per un universo specifico & COUNT, OPTIONAL \\
\hline
getCharactersByUniverse & Lista personaggi appartenenti a un universo & FILTER, rdf:type \\
\hline
getWorksByUniverse & Opere di un universo con tipo dedotto (Book/Movie/TVSeries) & BIND, EXISTS, IF \\
\hline
getLocationsByUniverse & Luoghi di un universo con tipo e funzione narrativa & STRAFTER, OPTIONAL \\
\hline
getEntityDetails & Tipo, label e descrizione di un'entità generica & COALESCE, subClassOf* \\
\hline
getCharacterRelations & Relazioni complete di un personaggio (alleati, nemici, oggetti, abilità) & OPTIONAL multipli \\
\hline
getLocationRelations & Relazioni di un luogo (opere ambientate, organizzazioni, gerarchia) & OPTIONAL multipli \\
\hline
getObjectRelations & Relazioni di un oggetto (possessori, abilità conferite, distruttibilità) & OPTIONAL \\
\hline
getWorkRelations & Relazioni di un'opera (personaggi, luoghi, prequel/sequel, adattamenti) & OPTIONAL multipli \\
\hline
\textbf{getMoviesFromWikidata} & \textbf{Film Harry Potter: combina dati locali con Wikidata (7 film, registi, durate)} & \textbf{SERVICE, UNION, VALUES, MIN, SAMPLE} \\
\hline
\textbf{getLotrMoviesFromWikidata} & \textbf{Film LOTR: combina dati locali con Wikidata (5 film, registi, durate)} & \textbf{SERVICE, UNION, VALUES, MIN, SAMPLE} \\
\hline
\textbf{getPercyJacksonMoviesFromWikidata} & \textbf{Film Percy Jackson: combina dati locali con Wikidata (1 film, regista, durata)} & \textbf{SERVICE, UNION, VALUES, MIN, SAMPLE} \\
\hline
\end{tabular}
\caption{Query SPARQL implementate}
\label{tab:queries}
\end{table}

\subsection*{Query Federata Esempio}

\begin{tcolorbox}[colback=white,colframe=black,title=\textbf{Query Federata: Film Harry Potter},arc=2mm,boxrule=0.8pt]
\begin{verbatim}
PREFIX ontology: <http://www.narrative-universes.org/ontology#>
PREFIX wd: <http://www.wikidata.org/entity/>
PREFIX wdt: <http://www.wikidata.org/prop/direct/>

SELECT ?film ?title ?releaseYear ?director ?source
WHERE {
  {
    # Dati locali
    ?film a ontology:Movie ;
          rdfs:label ?title ;
          ontology:belongsToUniverse :HarryPotterUniverse .
    BIND("Locale" AS ?source)
  }
  UNION
  {
    # Dati Wikidata
    SERVICE <https://query.wikidata.org/sparql> {
      VALUES ?film { wd:Q102244 wd:Q102448 wd:Q102225 }
      ?film rdfs:label ?title ;
            wdt:P577 ?releaseDate ;
            wdt:P57 ?directorEntity .
      ?directorEntity rdfs:label ?director .
      BIND(YEAR(?releaseDate) AS ?releaseYear)
      BIND("Wikidata" AS ?source)
    }
  }
}
\end{verbatim}
\end{tcolorbox}

\textbf{Risultati:} 8 film totali (1 locale + 7 Wikidata), con badge fonte nel frontend.

\subsection*{Applicazione Web}

\textbf{Architettura:} Frontend React + Proxy Node.js + GraphDB (LDP).

\textbf{Funzionalità principali:}
\begin{itemize}
    \item Homepage con 3 universi narrativi (Fig. \ref{fig:homepage})
    \item Dashboard universo con 4 tab: Personaggi, Luoghi, Opere, Dati Esterni (Fig. \ref{fig:dashboard})
    \item Pagine dettaglio entità con relazioni (Fig. \ref{fig:details})
    \item Integrazione Wikidata con badge fonte (Fig. \ref{fig:wikidata})
\end{itemize}

\textbf{Flow chart interazione:} Figura \ref{fig:flowchart} mostra percorso completo di navigazione dall'homepage all'integrazione Wikidata.

\begin{figure}[p]
    \centering
    \includegraphics[width=0.9\textwidth,height=0.95\textheight,keepaspectratio]{ui_webapp.pdf}
    \caption{Flow chart interazione applicazione}
    \label{fig:flowchart}
\end{figure}

\begin{figure}[p]
    \centering
    \includegraphics[width=0.9\textwidth]{Homepage_with_3_Universes.png}
    \caption{Homepage applicazione}
    \label{fig:homepage}
\end{figure}

\begin{figure}[p]
    \centering
    \includegraphics[width=0.9\textwidth]{Universe_Dashboard_HarryPotter.png}
    \caption{Dashboard universo}
    \label{fig:dashboard}
\end{figure}

\begin{figure}[p]
    \centering
    \includegraphics[width=0.9\textwidth]{Character_Details_Page.png}
    \caption{Dettagli personaggio}
    \label{fig:details}
\end{figure}

\begin{figure}[p]
    \centering
    \includegraphics[width=0.9\textwidth]{External_Data_Wikidata_Badge.png}
    \caption{Integrazione Wikidata}
    \label{fig:wikidata}
\end{figure}

\clearpage

% =================================================================
% PARTE 10: CONCLUSIONI
% =================================================================

\section*{Conclusioni}

Il progetto realizza un sistema completo per la rappresentazione semantica di universi narrativi, dimostrando l'applicabilità del Web Semantico a domini complessi.

\textbf{Risultati:} Ontologia OWL con 30+ classi, 20+ object properties, 11 data properties; knowledge graph con 60+ individui; 12 query SPARQL di cui 3 federate; applicazione React funzionante; allineamenti schema.org e Wikidata; 5 regole SWRL con inferenze materializzate.

\textbf{Competenze:} Modellazione OWL con pattern avanzati (enumerazioni, restrizioni, property chains); query SPARQL con aggregazioni e federazione; integrazione LDP; reasoning con SWRL.

Il modello costituisce una base solida per estensioni future (nuovi universi, ulteriori fonti LOD come DBpedia) e dimostra il potenziale del Web Semantico come infrastruttura distribuita di conoscenza.

\end{document}