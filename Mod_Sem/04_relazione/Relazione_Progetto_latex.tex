\documentclass[12pt]{article}
\usepackage[utf8]{inputenc}
\usepackage{tcolorbox}
\usepackage{amsmath}
\usepackage{hyperref}
\usepackage{geometry}
\usepackage{xcolor}
\usepackage{graphicx}
\geometry{margin=2.5cm}

\begin{document}

\thispagestyle{empty}

\begin{center}
    {\large \textsc{Università degli Studi di Torino}}\\
    {\large \textsc{Dipartimento di Informatica}}\\[0.8cm]
    {\textcolor{gray}{\large\href{https://informatica.i-learn.unito.it/course/view.php?id=3571}
    {Modellazione Concettuale del Web Semantico — a.a.\ 2025/2026}}}\\[2.2cm]
    {\Huge \textbf{Relazione del Progetto}}\\[0.8cm]
    {\LARGE \textbf{Narrative Universes}}\\[1.4cm]

    {\textcolor{gray}{\large Alessandro Olivero (matricola: 915069)\\
    Giovanni Grillo (matricola: 989819)}}
\end{center}

\vfill

\begin{center}
    {\textcolor{gray}{Gennaio 2026}}
\end{center}

\newpage





\section*{Introduzione}

Il presente progetto nasce all'interno del corso di \textit{Modellazione Concettuale del Web Semantico} e ha come obiettivo la progettazione, realizzazione e interrogazione di un'ontologia complessa, capace di rappresentare in modo formale e strutturato un insieme di universi narrativi. L'idea alla base del lavoro è quella di costruire un modello semantico che permetta di descrivere personaggi, luoghi, organizzazioni, opere e relazioni interne a uno o più universi, mantenendo al tempo stesso un elevato livello di generalità e riusabilità.

L'ontologia è stata sviluppata seguendo i principi del Web Semantico e utilizzando gli standard W3C, in particolare RDF, RDFS e OWL. La modellazione è accompagnata da un insieme di query SPARQL, progettate per esplorare la struttura del grafo, verificare la correttezza del modello e dimostrare la capacità dell'ontologia di rispondere a interrogazioni complesse. Una parte del progetto è inoltre dedicata all'integrazione con fonti esterne, come Wikidata, tramite federated queries.

\vspace{0.5cm}

\section*{Obiettivi del Progetto}

\begin{itemize}
    \item Progettare un'ontologia modulare e scalabile, capace di rappresentare universi narrativi differenti.
    \item Definire classi, proprietà e individui in modo coerente, evitando ridondanze e garantendo consistenza semantica.
    \item Modellare relazioni significative tra personaggi, luoghi, organizzazioni e opere.
    \item Implementare un insieme di query SPARQL in grado di esplorare il grafo e dimostrare la correttezza del modello.
    \item Integrare l'ontologia con risorse esterne tramite federated queries.
    \item Produrre una documentazione chiara, leggibile e strutturata.
\end{itemize}

\vspace{0.5cm}

\begin{tcolorbox}[
  colback=white,
  colframe=black,
  title=\textbf{Box 1: Descrizione Generale dell'Ontologia},
  fonttitle=\bfseries,
  arc=2mm,
  boxrule=0.8pt
]
L'ontologia rappresenta uno o più universi narrativi attraverso classi, proprietà e individui. Include entità come personaggi, luoghi, organizzazioni, opere e universi, collegate tramite relazioni semantiche significative. Il modello è estensibile, coerente e interrogabile tramite SPARQL, e costituisce la base per analisi semantiche e integrazioni con risorse esterne come Wikidata.
\end{tcolorbox}

\vspace{0.5cm}

\section*{Struttura Concettuale}

La struttura concettuale dell'ontologia è stata definita seguendo un approccio top-down: si è partiti dall'identificazione delle entità fondamentali e delle loro relazioni, per poi procedere alla definizione delle classi e delle proprietà. Il risultato è un modello gerarchico, in cui le classi principali sono organizzate in modo da garantire coerenza semantica e facilità di estensione.

La classe \texttt{Character} è una delle più centrali, poiché rappresenta gli attori principali delle narrazioni. Essa è collegata a luoghi, opere e organizzazioni tramite proprietà specifiche, permettendo di ricostruire percorsi narrativi, appartenenze e ruoli.

La classe \texttt{Location} permette di modellare ambienti fisici o concettuali, mentre \texttt{Organization} rappresenta gruppi, ordini o istituzioni. La classe \texttt{Work} consente di collegare entità narrative alle opere in cui compaiono, mentre \texttt{Universe} funge da contenitore semantico per raggruppare entità coerenti.

\vspace{0.5cm}

\begin{tcolorbox}[
  colback=white,
  colframe=black,
  title=\textbf{Box 2: Classi Principali},
  fonttitle=\bfseries,
  arc=2mm,
  boxrule=0.8pt
]
\begin{itemize}
    \item \textbf{Character}: individui che rappresentano personaggi appartenenti a uno o più universi.
    \item \textbf{Location}: luoghi fisici o concettuali rilevanti all'interno della narrazione.
    \item \textbf{Organization}: gruppi, istituzioni o entità a cui i personaggi possono appartenere.
    \item \textbf{Work}: opere letterarie, cinematografiche o artistiche.
    \item \textbf{Universe}: macro-strutture narrative che raggruppano entità coerenti.
\end{itemize}
\end{tcolorbox}

\vspace{0.5cm}

\begin{tcolorbox}[
  colback=white,
  colframe=black,
  title=\textbf{Box 3: Proprietà Principali},
  fonttitle=\bfseries,
  arc=2mm,
  boxrule=0.8pt
]
\begin{itemize}
    \item \texttt{appearsIn}: collega un personaggio o un luogo a un'opera.
    \item \texttt{belongsToUniverse}: collega un'entità all'universo narrativo di riferimento.
    \item \texttt{memberOf}: collega un personaggio a un'organizzazione.
    \item \texttt{mentors}: rappresenta relazioni di mentoring tra personaggi.
    \item \texttt{description}: fornisce una descrizione testuale dell'entità.
\end{itemize}
\end{tcolorbox}

\vspace{0.5cm}

\section*{Utilizzo di Protégé per la Modellazione}

La costruzione dell'ontologia è stata realizzata utilizzando \textbf{Protégé}, uno strumento fondamentale per la modellazione OWL. Protégé ha permesso di definire in modo visuale e strutturato le classi, le proprietà e gli individui che compongono l'ontologia degli universi narrativi.

Le principali attività svolte in Protégé includono:

\begin{itemize}
    \item \textbf{Definizione delle classi}: creazione delle classi principali e organizzazione della gerarchia.
    \item \textbf{Creazione delle proprietà}: definizione di proprietà oggetto e dato, con dominio e codominio.
    \item \textbf{Inserimento degli individui}: popolamento dell'ontologia con personaggi, luoghi, opere e organizzazioni.
    \item \textbf{Verifica con reasoner}: utilizzo di HermiT per controllare la consistenza del modello.
    \item \textbf{Esportazione RDF/XML}: esportazione dell'ontologia per l'importazione in GraphDB.
\end{itemize}

Protégé ha garantito una modellazione chiara, coerente e facilmente estendibile, facilitando la gestione dell'intero ciclo di sviluppo dell'ontologia.

\vspace{0.5cm}

\section*{Modellazione OWL}

La modellazione OWL ha permesso di definire formalmente le relazioni tra le entità. Le proprietà sono state arricchite con dominio, codominio e caratteristiche semantiche, permettendo inferenze automatiche tramite reasoner.

\vspace{0.5cm}

\begin{tcolorbox}[
  colback=white,
  colframe=black,
  title=\textbf{Box 4: Esempio di Proprietà OWL},
  fonttitle=\bfseries,
  arc=2mm,
  boxrule=0.8pt
]
La proprietà \texttt{memberOf} è stata modellata come \textit{ObjectProperty} con:
\begin{itemize}
    \item \textbf{Dominio}: \texttt{Character}
    \item \textbf{Codominio}: \texttt{Organization}
\end{itemize}
Essa rappresenta l'appartenenza di un personaggio a un gruppo o istituzione.
\end{tcolorbox}

\vspace{0.5cm}

\section*{Query SPARQL}

Le query SPARQL sono state utilizzate per esplorare il grafo e verificare la correttezza del modello.

\begin{tcolorbox}[
  colback=white,
  colframe=black,
  title=\textbf{Box 5: Query SPARQL -- Individui e Classi},
  fonttitle=\bfseries,
  arc=2mm,
  boxrule=0.8pt
]
\begin{verbatim}
SELECT DISTINCT ?individual ?label ?class
WHERE {
  ?individual rdf:type ?class ;
              rdfs:label ?label .
  FILTER(isIRI(?individual))
  FILTER(isIRI(?class))
}
ORDER BY ?class ?label
\end{verbatim}
\end{tcolorbox}

\vspace{0.5cm}

\section*{Integrazione con il Web Semantico}

L'integrazione con Wikidata tramite federated queries permette di arricchire il grafo locale con informazioni esterne, come descrizioni, date e collegamenti a entità globali.

\vspace{0.5cm}

\begin{tcolorbox}[
  colback=white,
  colframe=black,
  title=\textbf{Box 6: Integrazione con Wikidata},
  fonttitle=\bfseries,
  arc=2mm,
  boxrule=0.8pt
]
L'integrazione con Wikidata consente di estendere il grafo locale con dati provenienti dal Web Semantico, migliorando la ricchezza informativa dell'ontologia senza duplicare informazioni.
\end{tcolorbox}

\vspace{0.5cm}

\section*{Conclusioni}

Il progetto ha permesso di progettare e realizzare un'ontologia per la rappresentazione di universi narrativi, dimostrando come gli strumenti del Web Semantico possano essere utilizzati per modellare domini complessi in modo formale e interrogabile. L'integrazione con Wikidata evidenzia il potenziale del Web Semantico come infrastruttura distribuita di conoscenza. Il modello sviluppato costituisce una base solida per future estensioni e arricchimenti.

\newpage

\section{Query SPARQL e Applicazione Client}

\subsection{Flusso di Interazione Utente}

L'applicazione web sviluppata permette agli utenti di esplorare gli universi narrativi modellati nell'ontologia attraverso un'interfaccia intuitiva e interattiva. Il sistema è stato progettato per supportare tre tipologie di utenti:

\begin{itemize}
    \item \textbf{Appassionati di narrativa fantasy}: utenti interessati a esplorare personaggi, luoghi e opere di universi narrativi come Harry Potter, Percy Jackson e Il Signore degli Anelli.
    \item \textbf{Studenti e ricercatori}: utenti che necessitano di consultare informazioni strutturate e semanticamente arricchite per scopi accademici.
    \item \textbf{Sviluppatori e data scientist}: utenti che vogliono interrogare il knowledge graph tramite query SPARQL e integrare dati esterni.
\end{itemize}

\subsubsection{Task Principali}

\begin{enumerate}
    \item \textbf{Esplorazione degli universi narrativi}: visualizzare gli universi disponibili con statistiche aggregate (numero di personaggi, luoghi, opere).
    \item \textbf{Consultazione di entità}: navigare tra personaggi, luoghi e opere con possibilità di filtraggio per tipo.
    \item \textbf{Visualizzazione dettagli}: accedere a informazioni dettagliate su singole entità, incluse relazioni con altre entità (alleati, nemici, organizzazioni, abilità).
    \item \textbf{Arricchimento dati}: integrare informazioni da Wikidata tramite query federate per arricchire il knowledge graph locale.
\end{enumerate}

\subsection{Flow Chart dell'Interazione}

La Figura~\ref{fig:flowchart} illustra il flusso completo di navigazione nell'applicazione. L'utente parte dalla homepage, seleziona un universo narrativo, esplora le diverse sezioni tramite tab (Personaggi, Luoghi, Opere, Dati Esterni), e può arricchire i dati locali con informazioni da Wikidata tramite query federate.

\begin{figure}[p]
    \centering
    \includegraphics[width=\textwidth,height=0.95\textheight,keepaspectratio]{ui_webapp.pdf}
    \caption{Flow chart dell'interazione utente con l'applicazione web}
    \label{fig:flowchart}
\end{figure}

Il diagramma mostra chiaramente i percorsi di navigazione disponibili e le query SPARQL eseguite in ciascuna fase dell'interazione.

\clearpage

\subsection{Interfaccia Applicazione}

\subsubsection{Homepage}

La Figura~\ref{fig:homepage} mostra la pagina iniziale dell'applicazione, dove vengono visualizzati i tre universi narrativi disponibili (Harry Potter, Percy Jackson, Terra di Mezzo). Per ogni universo vengono mostrate statistiche aggregate ottenute tramite la query \texttt{getUniverses}, che utilizza \texttt{COUNT} e \texttt{GROUP BY} per contare personaggi, luoghi e opere.

\begin{figure}[h]
    \centering
    \includegraphics[width=\textwidth,height=0.85\textheight,keepaspectratio]{Homepage_with_3_Universes.png}
    \caption{Homepage con lista degli universi narrativi}
    \label{fig:homepage}
\end{figure}

\clearpage

\subsubsection{Dashboard Universo}

La Figura~\ref{fig:dashboard} mostra la dashboard di un universo (Harry Potter), con quattro tab di navigazione: Personaggi, Luoghi, Opere e Dati Esterni. La dashboard viene popolata tramite la query \texttt{getUniverseDetails}.

\begin{figure}[h]
    \centering
    \includegraphics[width=\textwidth,height=0.85\textheight,keepaspectratio]{Universe_Dashboard_HarryPotter.png}
    \caption{Dashboard dell'universo Harry Potter con 4 tab}
    \label{fig:dashboard}
\end{figure}

\clearpage

\subsubsection{Tab Opere}

La Figura~\ref{fig:works} mostra il tab delle opere narrative dell'universo Harry Potter. La query \texttt{getWorksByUniverse} utilizza \texttt{BIND}, \texttt{IF} ed \texttt{EXISTS} per recuperare le opere con informazioni su anno di pubblicazione e tipo (libro, film, ecc.).

\begin{figure}[h]
    \centering
    \includegraphics[width=\textwidth,height=0.85\textheight,keepaspectratio]{narrativeWork_pane.png}
    \caption{Tab Opere con lista delle opere narrative}
    \label{fig:works}
\end{figure}

\clearpage

\subsubsection{Dettagli Entità}

La Figura~\ref{fig:details} mostra la pagina di dettaglio di un personaggio (Harry Potter), con informazioni su alleati, nemici, oggetti posseduti, abilità e opere in cui appare. Le informazioni vengono recuperate tramite le query \texttt{getEntityDetails} e \texttt{getCharacterRelations}.

\begin{figure}[h]
    \centering
    \includegraphics[width=\textwidth,height=0.85\textheight,keepaspectratio]{Character_Details_Page.png}
    \caption{Pagina dettagli di un personaggio con relazioni}
    \label{fig:details}
\end{figure}

\clearpage

\subsubsection{Dati Esterni da Wikidata}

La Figura~\ref{fig:wikidata} mostra il tab "Dati Esterni" con i risultati delle query federate verso Wikidata. I film vengono mostrati con un badge che indica la fonte: "Locale" (verde) per i dati dal GraphDB locale, "Wikidata" (blu) per i dati ottenuti tramite query federata.

\begin{figure}[h]
    \centering
    \includegraphics[width=\textwidth,height=0.85\textheight,keepaspectratio]{External_Data_Wikidata_Badge.png}
    \caption{Tab Dati Esterni con badge Locale/Wikidata}
    \label{fig:wikidata}
\end{figure}

Questo dimostra l'integrazione tra dati locali e dati esterni, uno dei requisiti fondamentali del progetto.

\clearpage

\subsection{Query SPARQL Implementate}

Sono state implementate 12 query SPARQL, di cui 3 query federate verso Wikidata. La Tabella~\ref{tab:queries} riassume le query principali con descrizione e caratteristiche.

\begin{table}[h]
\centering
\small
\begin{tabular}{|p{4cm}|p{6cm}|p{3cm}|}
\hline
\textbf{Nome Query} & \textbf{Descrizione} & \textbf{Caratteristiche} \\
\hline
\texttt{getUniverses} & Recupera tutti gli universi con statistiche aggregate & COUNT, GROUP BY \\
\hline
\texttt{getUniverseDetails} & Dettagli di un universo specifico & COUNT, OPTIONAL \\
\hline
\texttt{getCharactersByUniverse} & Lista personaggi con tipi & FILTER, OPTIONAL \\
\hline
\texttt{getWorksByUniverse} & Lista opere con anno e tipo & BIND, IF, EXISTS \\
\hline
\texttt{getLocationsByUniverse} & Lista luoghi con funzioni narrative & OPTIONAL, FILTER \\
\hline
\texttt{getEntityDetails} & Dettagli completi di un'entità & COALESCE, OPTIONAL \\
\hline
\texttt{getCharacterRelations} & Relazioni di un personaggio & OPTIONAL multipli \\
\hline
\texttt{getLocationRelations} & Relazioni di un luogo & OPTIONAL multipli \\
\hline
\texttt{getObjectRelations} & Relazioni di un oggetto & OPTIONAL multipli \\
\hline
\texttt{getWorkRelations} & Relazioni di un'opera & OPTIONAL multipli \\
\hline
\textbf{getMoviesFromWikidata} & \textbf{Query federata film Harry Potter} & \textbf{SERVICE, UNION, GROUP BY} \\
\hline
\textbf{getLotrMoviesFromWikidata} & \textbf{Query federata film LOTR} & \textbf{SERVICE, UNION} \\
\hline
\textbf{getPercyMoviesFromWikidata} & \textbf{Query federata film Percy Jackson} & \textbf{SERVICE, UNION} \\
\hline
\end{tabular}
\caption{Query SPARQL implementate nell'applicazione}
\label{tab:queries}
\end{table}

\newpage

\subsection{Query Federate verso Wikidata}

Sono state implementate 3 query federate che interrogano l'endpoint SPARQL pubblico di Wikidata per arricchire i dati locali con informazioni esterne su film e registi.

\subsubsection{Architettura delle Query Federate}

Le query federate utilizzano la clausola \texttt{SERVICE} di SPARQL per interrogare Wikidata. La struttura delle query combina:

\begin{itemize}
    \item \textbf{Dati locali}: film presenti nel knowledge graph locale (GraphDB)
    \item \textbf{Dati esterni}: film recuperati da Wikidata tramite ID specifici (Q-codes)
    \item \textbf{Unione}: \texttt{UNION} per combinare i due set di dati
    \item \textbf{Aggregazione}: \texttt{GROUP BY} per eliminare duplicati
\end{itemize}

\subsubsection{Esempio: Query Federata Film Harry Potter}

La query \texttt{getMoviesFromWikidata} combina i film di Harry Potter presenti localmente con quelli recuperati da Wikidata:

\begin{tcolorbox}[
  colback=white,
  colframe=black,
  title=\textbf{Query Federata: Film Harry Potter},
  fonttitle=\bfseries,
  arc=2mm,
  boxrule=0.8pt
]
\begin{verbatim}
PREFIX ontology: <http://www.narrative-universes.org/ontology#>
PREFIX rdfs: <http://www.w3.org/2000/01/rdf-schema#>
PREFIX wd: <http://www.wikidata.org/entity/>
PREFIX wdt: <http://www.wikidata.org/prop/direct/>

SELECT ?film ?title (MIN(?year) AS ?releaseYear) 
       (SAMPLE(?directorName) AS ?director) 
       (SAMPLE(?runtime) AS ?duration) ?source
WHERE {
  {
    ?film a ontology:Movie ;
          rdfs:label ?title ;
          ontology:belongsToUniverse <UNIVERSE_URI> .
    OPTIONAL { ?film ontology:publicationYear ?year }
    OPTIONAL { ?film ontology:runtime ?runtime }
    BIND("Locale" AS ?source)
    BIND("" AS ?directorName)
  }
  UNION
  {
    SERVICE <https://query.wikidata.org/sparql> {
      VALUES ?wikidataFilm {
        wd:Q102244 wd:Q102448 wd:Q102225 
        wd:Q102235 wd:Q161687 wd:Q161678 wd:Q232009
      }
      ?wikidataFilm rdfs:label ?title ;
                    wdt:P577 ?releaseDate ;
                    wdt:P57 ?directorEntity .
      OPTIONAL { ?wikidataFilm wdt:P2047 ?runtime }
      ?directorEntity rdfs:label ?directorName .
      FILTER(LANG(?title) = "en")
      FILTER(LANG(?directorName) = "en")
      BIND(YEAR(?releaseDate) AS ?year)
      BIND(?wikidataFilm AS ?film)
      BIND("Wikidata" AS ?source)
    }
  }
}
GROUP BY ?film ?title ?source
ORDER BY ?releaseYear
\end{verbatim}
\end{tcolorbox}

\subsubsection{Risultati delle Query Federate}

Le query federate restituiscono 8 film per Harry Potter (1 locale + 7 da Wikidata), 6 film per Il Signore degli Anelli (1 locale + 5 da Wikidata), e 2 film per Percy Jackson (1 locale + 1 da Wikidata).

Ogni risultato include:
\begin{itemize}
    \item Titolo del film
    \item Anno di uscita
    \item Regista (solo per film da Wikidata)
    \item Durata in minuti
    \item Fonte (badge "Locale" o "Wikidata")
\end{itemize}

\newpage

\subsection{Esempio di Interazione Completa}

\textbf{Scenario:} Un utente vuole esplorare i personaggi dell'universo Harry Potter e arricchire le informazioni con dati esterni da Wikidata.

\subsubsection{Passi dell'Interazione}

\begin{enumerate}
    \item L'utente apre l'applicazione all'indirizzo \texttt{http://localhost:3000}
    \item Viene visualizzata la homepage con 3 card degli universi (query: \texttt{getUniverses})
    \item L'utente clicca su "Esplora Universo" di Harry Potter
    \item Si apre la dashboard con 4 tab e statistiche dell'universo (query: \texttt{getUniverseDetails})
    \item L'utente seleziona il tab "Personaggi"
    \item Viene eseguita la query \texttt{getCharactersByUniverse} che restituisce 17 personaggi
    \item Sono visibili filtri per tipo: Protagonisti (3), Alleati (5), Antagonisti (4), Mentori (2)
    \item L'utente clicca sul filtro "Protagonisti"
    \item Vengono mostrati solo Harry Potter, Hermione Granger e Ron Weasley
    \item L'utente clicca su "Harry Potter"
    \item Viene eseguita la query \texttt{getEntityDetails} seguita da \texttt{getCharacterRelations}
    \item Vengono visualizzati:
    \begin{itemize}
        \item Alleati: Ron Weasley, Hermione Granger, Albus Dumbledore
        \item Nemici: Lord Voldemort, Draco Malfoy
        \item Oggetti posseduti: Bacchetta di agrifoglio, Mantello dell'invisibilità
        \item Abilità: Patronus (cervo), Parseltongue
        \item Opere: Harry Potter e la Pietra Filosofale, Harry Potter e la Camera dei Segreti, ecc.
    \end{itemize}
    \item L'utente clicca sul pulsante "Torna indietro"
    \item Torna alla dashboard
    \item L'utente seleziona il tab "Dati Esterni"
    \item Clicca sul pulsante "Carica dati da Wikidata"
    \item Vengono eseguite in parallelo le query federate \texttt{getMoviesFromWikidata} e \texttt{getCharactersFromWikidata}
    \item Dopo 3-5 secondi, vengono visualizzati:
    \begin{itemize}
        \item \textbf{Sezione Film}: 8 film totali
        \begin{itemize}
            \item 1 film con badge verde "Locale": Harry Potter e la Pietra Filosofale (152 min, 2001)
            \item 7 film con badge blu "Wikidata": Camera dei Segreti (Chris Columbus, 2002), Prigioniero di Azkaban (Alfonso Cuarón, 2004), Calice di Fuoco (Mike Newell, 2005), ecc.
        \end{itemize}
        \item \textbf{Sezione Personaggi}: personaggi arricchiti con descrizioni da Wikidata (quando disponibili)
    \end{itemize}
\end{enumerate}

Questo esempio dimostra il flusso completo di interazione e l'integrazione tra dati locali e dati esterni tramite query federate.

\newpage

\subsection{Architettura Applicazione}

L'applicazione è strutturata su tre livelli:

\begin{itemize}
    \item \textbf{Frontend}: React.js per l'interfaccia utente
    \item \textbf{Proxy Server}: Node.js Express per gestire le richieste SPARQL e risolvere problemi CORS
    \item \textbf{Backend}: GraphDB come Linked Data Platform per memorizzare e interrogare l'ontologia
\end{itemize}

\subsubsection{Flusso delle Richieste}

\begin{enumerate}
    \item L'utente interagisce con l'interfaccia React
    \item React invia richieste HTTP al proxy Node.js (porta 3001)
    \item Il proxy inoltra le query SPARQL a GraphDB (porta 7200)
    \item Per le query federate, GraphDB interroga Wikidata tramite la clausola SERVICE
    \item I risultati vengono restituiti al proxy, poi a React
    \item React renderizza i dati nell'interfaccia
\end{enumerate}

\subsection{Tecnologie Utilizzate}

\begin{itemize}
    \item \textbf{Protégé 5.6}: modellazione ontologia OWL
    \item \textbf{GraphDB Free 10.7}: Linked Data Platform e SPARQL endpoint
    \item \textbf{React 18}: framework frontend
    \item \textbf{Node.js + Express}: proxy server
    \item \textbf{SPARQL 1.1}: linguaggio di query con supporto query federate
    \item \textbf{Wikidata Query Service}: endpoint pubblico per dati esterni
\end{itemize}

\newpage

\section*{Conclusioni Finali}

Il progetto ha permesso di progettare e realizzare un sistema completo per la rappresentazione semantica di universi narrativi, dimostrando come gli strumenti del Web Semantico possano essere utilizzati per modellare domini complessi in modo formale, interrogabile e interoperabile.

\textbf{Risultati principali:}
\begin{itemize}
    \item Ontologia OWL completa con 16+ classi, 8+ proprietà oggetto, 4+ data properties
    \item Knowledge graph popolato con 50+ individui di 3 universi narrativi
    \item 12 query SPARQL, di cui 3 query federate verso Wikidata
    \item Applicazione web React funzionante con interfaccia intuitiva
    \item Integrazione dati locali + dati esterni tramite SERVICE clause
\end{itemize}

\textbf{Competenze dimostrate:}
\begin{itemize}
    \item Modellazione concettuale con OWL e Protégé
    \item Progettazione query SPARQL con aggregazioni (COUNT, GROUP BY, SAMPLE, MIN)
    \item Query federate verso endpoint pubblici (Wikidata)
    \item Sviluppo applicazione client con Linked Data Platform
    \item Integrazione frontend-backend per sistemi semantici
\end{itemize}

Il modello sviluppato costituisce una base solida per future estensioni, come l'aggiunta di nuovi universi narrativi, l'arricchimento con dati da altre fonti LOD (DBpedia, VIAF), e l'implementazione di funzionalità di reasoning avanzate.

L'integrazione con Wikidata evidenzia il potenziale del Web Semantico come infrastruttura distribuita di conoscenza, dove dati locali e globali possono coesistere e arricchirsi reciprocamente attraverso standard aperti e interoperabili.

\end{document}